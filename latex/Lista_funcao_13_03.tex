\documentclass{article}
\usepackage[utf8]{inputenc}

\title{Cálculo Diferencial Lista 13.03.2020}
\author{Henrique Honório da Silva }
\date{March 2020}

\begin{document}

\maketitle

	\section{Considere os dados abaixo e escreva-os utilizando notação intervalar e notação de conjunto:}

		\hspace{5mm} (a)Todos os números \underline{reais maiores que 1};

			\hspace{5mm} \textbf{Intervalar:} $(1,+\infty)$

			\hspace{5mm}  \textbf{Conjunto:} $\{x \in \rm I\!R | 1 < x \}$

		\vspace{5mm}
		(b)Todos os números \underline{reais menores que -1};

			\hspace{5mm} \textbf{Intervalar:} $(-\infty,-1)$

			\hspace{5mm}  \textbf{Conjunto:} $\{x \in \rm I\!R | x < -1 \}$

		\vspace{5mm}
		(c)Todos os números \underline{reais menores que 2 e maiores que -1};

			\hspace{5mm} \textbf{Intervalar:} $(-1, 2)$

			\hspace{5mm}  \textbf{Conjunto:} $\{x \in \rm I\!R | -1 < x < 2 \}$

		\vspace{5mm}
		(d)Todos os números \underline{reais menores ou iguais a 0,2 e maiores ou iguais à -1};

			\hspace{5mm} \textbf{Intervalar:} $[-1, 0,2]$

			\hspace{5mm}  \textbf{Conjunto:} $\{x \in \rm I\!R | -1 \leq x \leq 0,2 \}$

		\vspace{5mm}
		(e)Todos os números \underline{reais maiores ou iguais à -3};

			\hspace{5mm} \textbf{Intervalar:} $[-3, +\infty)$

			\hspace{5mm}  \textbf{Conjunto:} $\{x \in \rm I\!R | -3 \leq x \}$

		\vspace{5mm}
		(f)Todos os números \underline{reais menores ou iguais à $-\sqrt{2}$};

			\hspace{5mm} \textbf{Intervalar:} $(-\infty,-\sqrt{2}]$

			\hspace{5mm}  \textbf{Conjunto:} $\{x \in \rm I\!R | x \leq -\sqrt{2} \}$

	\section{Dado duas funções quaisquer, pergunta-se quando o produto ou quociente dos mesmos é positivo? E quando é negativo?}

		\hspace{5mm}Remetemos às regras de sinais da multiplicação e divisão entre números reais: sinais iguais resultam em positivo, enquanto sinais diferentes implicam em negativo.

		Sendo correto afirmar que sempre que os sinais forem iquais a função será positiva, e sempre que os sinais forem
		diferentes a função será negativa.

		\vspace{5mm}
		Por exemplo, considerando a inequação

		$(x - 2).(x + 3) \geq 0$

		tomamos as funções

		$f(x) = x - 2$
		$f(x) = 0 \Leftrightarrow x - 2 = 0$

		\hspace{5mm} $\Rightarrow x = 2$

		e

		$g(x) = x + 3 \Leftrightarrow x - 3 = 0$

		\hspace{5mm} $\Rightarrow x = -3$

	\section{Considerando as respostas dos dois exercícios anteriores, obtenha a solução das inequações e escreva-os na notação intervalar e notação de conjunto:}

		\hspace{4mm} (a) $ x - 1 > 0 \Rightarrow x > 1$

		\hspace{10mm}ou

		\hspace{5mm} $ x - 1 < 0 \Rightarrow x < 1$

		\hspace{5mm} \textbf{Intervalar:} $]1[$

		\hspace{5mm} \textbf{Conjunto:} $ S = \{x \in \rm I\!R | 1 < x ou x < 1\}$

		\vspace{5mm}
		(b) $ x - 1 \geq 4$;

		\hspace{5mm}$ \Rightarrow x \geq 5$

		\hspace{5mm} \textbf{Intervalar:} $[5, +\infty)$

		\hspace{5mm} \textbf{Conjunto:} $ S = \{x \in \rm I\!R | 5 \leq x \}$

		\vspace{5mm}
		(c) $ x - 1 \leq - 3$;

		\hspace{5mm}$ \Rightarrow x \leq -2$

		\hspace{5mm} \textbf{Intervalar:} $ (-\infty, -2]$

		\hspace{5mm} \textbf{Conjunto:} $ S = \{x \in \rm I\!R | x \leq -2 \}$

		\vspace{5mm}
		(d) $\frac {x - 1}{x + 2} > 0$;

		\hspace{5mm}$ f(x) = x - 1$

		\hspace{5mm}$ \Rightarrow x - 1 = 0 $

		\hspace{5mm}$ \Rightarrow x = 1$

		\hspace{5mm}$ g(x) = x + 2 $

		\hspace{5mm}$ \Rightarrow x + 2 \neq 0 $

		\hspace{5mm}$ \Rightarrow x \neq -2$

		\hspace{5mm} \textbf{Intervalar:} $(-2, 1)$

		\hspace{5mm} \textbf{Conjunto:} $ S = \{x \in \rm I\!R | -2 < x < 1 \}$

		\vspace{5mm}
		(e) $(x - 1)(x + 2) \leq 0$;

		\hspace{5mm}$ f(x) = x - 1$

		\hspace{5mm}$ \Rightarrow x - 1 = 0 $

		\hspace{5mm}$ \Rightarrow x = 1$

		\hspace{5mm}$ g(x) = x + 2$

		\hspace{5mm}$ \Rightarrow x + 2 = 0 $

		\hspace{5mm}$ \Rightarrow x = -2$

		\hspace{5mm} \textbf{Intervalar:} $[-2, 1]$

		\hspace{5mm} \textbf{Conjunto:} $ S = \{x \in \rm I\!R | -2 \leq x \leq 1 \}$

		\vspace{5mm}
		(f) $\frac {x - 1}{x + 2} \geq 0$;

		\hspace{5mm}$ f(x) = x - 1$

		\hspace{5mm}$ \Rightarrow x - 1 = 0 $

		\hspace{5mm}$ \Rightarrow x = 1$

		\hspace{5mm}$ g(x) = x + 2$

		\hspace{5mm}$ \Rightarrow x + 2 \neq 0 $

		\hspace{5mm}$ \Rightarrow x \neq -2$

		\hspace{5mm} \textbf{Intervalar:} $(-2, 1]$

		\hspace{5mm} \textbf{Conjunto:} $ S = \{x \in \rm I\!R | -2 < x \leq 1 \}$

	\section{Obtenha uma função $f(x) = ax + b$, satisfazendo as condições dadas $f(-5) = -1, f(2) = 4$.}

		\hspace{5mm} $f(x) = ax + b \Leftrightarrow f(-5) = -1$

		$\Rightarrow -5a + b = -1$

		$\Rightarrow -5a = -1 - b$

		\fbox{$\Rightarrow a = \frac{-1 -b}{5}$}

		$\Rightarrow -5.\frac{(-1 - b)}{5} + b= -1$

		$\Rightarrow -\not{5}.\frac{(-1 - b)}{\not{5}} + b= -1$

		$\Rightarrow (-1 - b) + b = -1$

		$\Rightarrow -b - 2b = -1$

		$\Rightarrow  -3b = -1$

		\fbox{$\Rightarrow b = \frac{-1}{3}$}

		$\Rightarrow a = \frac{-1 - (\frac{-1}{3})}{5}$

	\section{Determine a inclinação, a intersecção com o eixo x e y e esboce o gráfico de $2y + 3x = 0$.}

	\section{Escreva as equações para:}

		\hspace{4mm} (a) A inclinação é 5 e intercepta o eixo y no ponto (0, -4);

		(b) A inclinação é -2 e passa pelo ponto (1, 3);

		(c) Intercepta o eixo x no ponto (3, 0) e o eixo y no ponto (0, -2/3.

		(d) Passa pelo ponto (5, 4) é e paralela à reta 2x + y = 3.


	\section{Determine o valor de c para o qual a curva $y = 3x^2 - 2x + c$, passa pelo ponto (2, 4).}


	\section{ Obtenha uma função quadrática tal que $f(-1) = -4, f(1) = 2, f(2)= -1$.}


\end{document}
